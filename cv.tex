%______________________________________________________________________________________________________________________
% @brief    LaTeX2e Resume for Aaron R Quinlan
\documentclass[margin,line]{cv}
\usepackage{url}
\usepackage{enumerate}

%______________________________________________________________________________________________________________________
\begin{document}
\name{\Large Yuankun Xue}
\begin{resume}
    %__________________________________________________________________________________________________________________
    % Contact Information
    \section{\mysidestyle Contact\\Information}
    Ph.D. candidate                                                                     \hfill (Cell): 213.400.8361\\%
    Department of Electrical Engineering                                                    \hfill (Email): \url{yuankunx@usc.edu}\\%
    Viterbi School of Engineering                       \hfill    \url{http://yuankunx.wixsite.com/yuankun}\\%
    University of Southern California                                                                                       \hfill \\%

    %__________________________________________________________________________________________________________________
    % Research Interests
    \section{\mysidestyle Technical\\Expertise}

    Algorithm design, Machine learning, Mathematical optimization and software development, Complex Network, Time Series Analysis, Information Theory, Bioinformatics, Computer Architecture
    
    %__________________________________________________________________________________________________________________
    % Development Language
    \section{\mysidestyle Development Languages}

    Python, C/C++, Matlab, R, Linux Shell, Perl, Verilog
%__________________________________________________________________________________________________________________
    % Education
    \section{\mysidestyle Education}

    \textbf{University of Southern California}, Los Angeles, CA, USA\\
    Ph.D. candidate, Electrical Engineering, 2014 - now\\
    Overall GPA: 3.91/4
    
    \textbf{Fudan University}, Shanghai, China\\
    M.S., Electrical Engineering, 2014, Graduated with Highest honor
    
    \textbf{Fudan University}, Shanghai, China\\
    B.S., Electrical Engineering, 2011 \\


    %__________________________________________________________________________________________________________________
    % Publications
    \section{\mysidestyle Selected \\ Projects}
\textbf{Design and Optimization of NoC-based Systems For Cyber-physical System(CPS)}  \\ \textit{Jan 2014 - Present}\\
\textit{Published \textbf{8} papers(including \textbf{5} first-author papers)} all on top-tier conferences and journals like \textbf{DAC, CODES+ISSS, ICCAD, TVLSI and NOCS.} \\
\textbf{1.} Developed the first large scale Networks-on-chip (NoC) based manycore accelerator for Protein Folding Simulation that has achieved near-linear speedup (\textbf{DAC 2014}).\\
\textbf{2.} Proposed network bandwidth and resource time-multiplexing approaches for hierarchical parallel genetic algorithm (HPGA) accelerator (\textbf{NOCS 2014}).\\
\textbf{3.} Proposed a user-cooperation based linear network-coding scheme to improve significantly the throughput of collective communication of NoC-based manycore system (\textbf{NOCS 2015}).\\
\textbf{4.} Developed a general mathematical optimization framework for automatic synthesis of on-chip network topology with guaranteed performance bound (\textbf{NOCS 2016}).\\
\textbf{5.} Proposed a complex-network theory based scalable and realistic benchmark synthesis tool using LLVM for performance evaluation of manycore system (\textbf{CODES+ISSS 2016}).

\textbf{Machine-learning based Mathematical Modeling and Analysis of Complex Systems}  \\ \textit{July 2015 - Present}\\
\textit{Published \textbf{4} first-author papers} on top-tier conferences and journals including \textbf{DATE, Allerton Conference, ICCPS, TODAES}

\textbf{1.}  Proposed a spatio-temporal fractal dynamical system model capturing the inter-dependencies of muscles involved in forearm movements. Develop a multi-regression algorithm for parameter estimation.(\textbf{DATE 2016})\\
\textbf{2.}  Proposed a mathematical framework investigating the minimum number of sensors to ensure Observability of physiological systems. Proved the feasibility space of the problem is submodular and proposed a greedy-algorithm that delivers solutions with guaranteed optimality. (\textbf{Allerton Conference 2016})\\
\textbf{3.}  Proposed a dynamical graph model for internet-of-things (IoT) that enables the investigation of a set of under-explored key challenges in IoT domain (\textbf{TODAES, accepted to appear}\\
\textbf{4.} Proposed a new mathematical strategy for constructing compact yet accurate fractional-order non-linear models of complex systems dynamics that aim to scrutinize the causal effects and influences by analyzing the statistics of the magnitude increments and the inter-event times of stochastic processes (\textbf{ICCPS}, to appear). 

%__________________________________________________________________________________________________________________
    % Manuscript Review
    \section{\mysidestyle Journal Review}
    Guest reviewer for:\\
        \textit{Nature Scientific Report}, \textit{IMA Journal of Mathematical Control and Information}, \textit{IEEE Transactions on Very Large Scale Integration Systems}, \textit{Hindawi Mathematical Problems in Engineering}
    %__________________________________________________________________________________________________________________
    % Honours and Awards
    \section{\mysidestyle Honors and\\Awards}

    Student travel grant of Networks-on-chip Symposium (NOCS 2015). \\\vspace{1mm}%
    Excellent Graduate of Shanghai 2014.                                                  %



    %__________________________________________________________________________________________________________________
    % Publications
    \section{\mysidestyle Publications}



    \textbf{Conference} \\

    \begin{list}{*}{}
    
    \item [13.] \textbf{Yuankun Xue} and Paul Bogdan, Constructing Compact Causal Mathematical Models for Complex Dynamics​, (to appear in) Proceedings of 8th ACM/IEEE International Conference on Cyber-Physical System (ICCPS), 2017.
    
    \item [12.] \textbf{Yuankun Xue}, Sergio Pequito, Joana Maria Rosado Coelho, Paul Bogdan, George Pappas, Minimum Number of Sensors to Ensure Observability of Physiological Systems: a Case Study, (to appear in) Proceedings of 54th Annual Allerton Conference on Communication, Control, and Computing (Allerton), 2016.
    
\item [11.] Xue Lin, \textbf{Yuankun Xue}, Paul Bogdan, Massoud Pedram, Yanzhi Wang and Siddarth Garg, "Power-aware virtual machine mapping in the data-center-on-a-chip paradigm," in Proc. of the 34nd IEEE International Conference on Computer Design (ICCD), Oct. 2016. 

\item [10.] \textbf{Yuankun Xue}, Paul Bogdan, Scalable and Realistic Benchmark Synthesis for Efficient NoC Performance Evaluation: A Complex Network Analysis Approach, (to appear in) Proceedings of the International Conference on Hardware/Software Codesign and System Synthesis (CODES+ISSS), 2016

\item [9.] \textbf{Yuankun Xue}, Paul Bogdan, Improving NoC Performance under Spatio-Temporal Variability By Runtime Reconfiguration: A General Mathematical Framework, (to appear in) Proceedings of the 10th International Symposium on Networks-on-Chip (NOCS), 2016

\item [8.] \textbf{Yuankun Xue}, Saul Rodriguez, Paul Bogdan, A Spatio-Temporal Fractal Model for a CPS Approach to Brain-Machine-Body Interfaces, Design, Automation and Test in Europe Conference and Exhibition (DATE), 2016.        
        
\item [7.] \textbf{Yuankun Xue}, Paul Bogdan, User Cooperation Network Coding Approach for NoC Performance Improvement, Proceedings of the 9th International Symposium on Networks-on-Chip (NOCS), 2015.

\item [6.] Paul Bogdan, \textbf{Yuankun Xue}, Mathematical Models and Control Algorithms for Dynamic Optimization of Multicore Platforms: A Complex Dynamics Approach, International Conference On Computer Aided Design (ICCAD), 2015.

\item [5.] Paul Bogdan, Turbo Majumder, Arvind Ramanathan, \textbf{Yuankun Xue}, NoC Architectures as Enablers of Biological Discovery for Personalized and Precision Medicine, Proceedings of the 9th International Symposium on Networks-on-Chip (NOCS), 2015.

\item [4.] Paul Bogdan, \textbf{Yuankun Xue}, Cyber-physical systems for personalized and precise medicine,  2015 IEEE 58th International Midwest Symposium on Circuits and Systems (MWSCAS), 2015.

\item [3.] Alireza Shafaei, Yanzhi Wang, \textbf{Yuankun Xue}, Srikanth Ramadurgam, Paul Bogdan, Massoud Pedram, Prediction of the dark silicon phenomenon under deeply-scaled FinFET technologies, Proceedings of Great Lakes Symposium on VLSI (GLS-VLSI), 2015.

\item [2.] \textbf{Yuankun Xue}, Zhiliang Qian, Guopeng Wei, Paul Bogdan, Chi-Ying Tsui, Radu Marculescu, An efficient network-on-chip (noc) based multicore platform for hierarchical parallel genetic algorithms,  2014 Eighth IEEE/ACM International Symposium on Networks-on-Chip  (NOCS), 2014.

\item [1.] \textbf{Yuankun Xue}, Zhiliang Qian, Paul Bogdan, Fan Ye, Chi-Ying Tsui, Disease diagnosis-on-a-chip: Large scale networks-on-chip based multicore platform for protein folding analysis, Proceedings of the 51st Annual Design Automation Conference (DAC), 2014
    
    \end{list}
    \textbf{Journal} \\
    \begin{list}{*}{}
    \item [2.] \textbf{Yuankun Xue}, Ji Li, Shahin Nazarian and Paul Bogdan, Fundamental Challenges Towards Making IoT a Reachable Reality: A Model-centric Investigation, IEEE Transactions on Design Automation of Electronic Systems (TODAES), 2016 (to appear).
    
\item [1.] Karthi Duraisamy, \textbf{Yuankun Xue}, Paul Bogdan, Partha Pratim Pande, Multicast-Aware High-Performance Wireless Network-on-Chip Architectures, IEEE Transactions on Very Large Scale Integration (VLSI) Systems (TVLSI), 2016.
    	
    \end{list}
    %__________________________________________________________________________________________________________________
    % Seminars
    \section{\mysidestyle Teaching\\Experience}

    TA for EE499 Embedded Systems \hfill  2014 Fall
    
TA for EE454 Introduction to System-on-Chip \hfill 2015 Spring, 2015 Fall, 2016 Spring \\

    %__________________________________________________________________________________________________________________
    % Academic Service
    %__________________________________________________________________________________________________________________
    \section{\mysidestyle Seminar\\Talk}
    
    Data-Centers-on-a-Chip as Enablers for Cyber-Physical Systems: A Scalable Model of Computation Guiding the Design Methodologies of Network-on-Chip based Manycore Platforms\\
    \textit{Cyber-Physical Systems Seminar Series}, University of Southern California\\
    September 26, 2016

    %________________________________________________________________
    
    
    %____________________________________________________________
    \section{\mysidestyle Languages\\}
    Chinese (Native), English (Full working proficiency), Japanese (Full working proficiency)
\end{resume}
\end{document}


%______________________________________________________________________________
% EOF
